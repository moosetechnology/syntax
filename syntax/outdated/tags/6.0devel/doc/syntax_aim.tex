\documentclass[11pt]{article}
\usepackage{fullpage}

\begin{document}

The SYNTAX system is a set of tools whose first aim is to ease the design and
the implementation of the front-end part of translators --- mainly, but not
exclusive, in the compilation field.  These tools allow on one side to generate
analysers (scanner, parser and semantic) and, on the other side, to compile
source texts with these analysers.

Thus the goals of SYNTAX are identical with those of LEX and YACC.  But SYNTAX
is more powerful, especially when error processing is considered.  It possess
an automatic (though tunable) error repair and recovery mechanism.  Moreover,
though semantics can be processed by several means of its own (action, abstract
tree, \ldots{}) it can be coupled with more sophisticated ones as the FNC-2
attribute grammar system.

The main modules of SYNTAX are:

\begin{itemize}
\item {\sc Bnf}: takes as input a Context-Free Grammar (CFG) and produces an internal
      form used by the other modules.  Predicates can be specified in order to
      act upon the parsing process allowing to define non-deterministic or even
      context-sensitive (unambiguous) languages.

\item {\sc Csynt}: checks that the input CFG is LALR(1).  Conflicts can
      be resolved either by the user via a {\sc Prio} specification or by the
      system itself.

\item {\sc Lecl}: is the SYNTAX scanner generator.  Its input is the lexical
      specification level in regular expression form.  Predefined or user
      predicates together with an unbounded number of characters of lookahead
      can be used when needed.

\item {\sc Recor}: takes as input the error processing specifications.  A
      standard input is available.

\item {\sc Tables-C}: collects the outputs of the previous modules and
      generates a C program which must be linked with the SYNTAX library.

\end{itemize}



\end{document}

